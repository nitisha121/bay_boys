\documentclass[]{article}
\usepackage{lmodern}
\usepackage{amssymb,amsmath}
\usepackage{ifxetex,ifluatex}
\usepackage{fixltx2e} % provides \textsubscript
\ifnum 0\ifxetex 1\fi\ifluatex 1\fi=0 % if pdftex
  \usepackage[T1]{fontenc}
  \usepackage[utf8]{inputenc}
\else % if luatex or xelatex
  \ifxetex
    \usepackage{mathspec}
  \else
    \usepackage{fontspec}
  \fi
  \defaultfontfeatures{Ligatures=TeX,Scale=MatchLowercase}
\fi
% use upquote if available, for straight quotes in verbatim environments
\IfFileExists{upquote.sty}{\usepackage{upquote}}{}
% use microtype if available
\IfFileExists{microtype.sty}{%
\usepackage{microtype}
\UseMicrotypeSet[protrusion]{basicmath} % disable protrusion for tt fonts
}{}
\usepackage[margin=1in]{geometry}
\usepackage{hyperref}
\hypersetup{unicode=true,
            pdftitle={STAT 331 Final Report},
            pdfauthor={Nitisha Agarwal and Maya Perelman},
            pdfborder={0 0 0},
            breaklinks=true}
\urlstyle{same}  % don't use monospace font for urls
\usepackage{color}
\usepackage{fancyvrb}
\newcommand{\VerbBar}{|}
\newcommand{\VERB}{\Verb[commandchars=\\\{\}]}
\DefineVerbatimEnvironment{Highlighting}{Verbatim}{commandchars=\\\{\}}
% Add ',fontsize=\small' for more characters per line
\usepackage{framed}
\definecolor{shadecolor}{RGB}{248,248,248}
\newenvironment{Shaded}{\begin{snugshade}}{\end{snugshade}}
\newcommand{\KeywordTok}[1]{\textcolor[rgb]{0.13,0.29,0.53}{\textbf{#1}}}
\newcommand{\DataTypeTok}[1]{\textcolor[rgb]{0.13,0.29,0.53}{#1}}
\newcommand{\DecValTok}[1]{\textcolor[rgb]{0.00,0.00,0.81}{#1}}
\newcommand{\BaseNTok}[1]{\textcolor[rgb]{0.00,0.00,0.81}{#1}}
\newcommand{\FloatTok}[1]{\textcolor[rgb]{0.00,0.00,0.81}{#1}}
\newcommand{\ConstantTok}[1]{\textcolor[rgb]{0.00,0.00,0.00}{#1}}
\newcommand{\CharTok}[1]{\textcolor[rgb]{0.31,0.60,0.02}{#1}}
\newcommand{\SpecialCharTok}[1]{\textcolor[rgb]{0.00,0.00,0.00}{#1}}
\newcommand{\StringTok}[1]{\textcolor[rgb]{0.31,0.60,0.02}{#1}}
\newcommand{\VerbatimStringTok}[1]{\textcolor[rgb]{0.31,0.60,0.02}{#1}}
\newcommand{\SpecialStringTok}[1]{\textcolor[rgb]{0.31,0.60,0.02}{#1}}
\newcommand{\ImportTok}[1]{#1}
\newcommand{\CommentTok}[1]{\textcolor[rgb]{0.56,0.35,0.01}{\textit{#1}}}
\newcommand{\DocumentationTok}[1]{\textcolor[rgb]{0.56,0.35,0.01}{\textbf{\textit{#1}}}}
\newcommand{\AnnotationTok}[1]{\textcolor[rgb]{0.56,0.35,0.01}{\textbf{\textit{#1}}}}
\newcommand{\CommentVarTok}[1]{\textcolor[rgb]{0.56,0.35,0.01}{\textbf{\textit{#1}}}}
\newcommand{\OtherTok}[1]{\textcolor[rgb]{0.56,0.35,0.01}{#1}}
\newcommand{\FunctionTok}[1]{\textcolor[rgb]{0.00,0.00,0.00}{#1}}
\newcommand{\VariableTok}[1]{\textcolor[rgb]{0.00,0.00,0.00}{#1}}
\newcommand{\ControlFlowTok}[1]{\textcolor[rgb]{0.13,0.29,0.53}{\textbf{#1}}}
\newcommand{\OperatorTok}[1]{\textcolor[rgb]{0.81,0.36,0.00}{\textbf{#1}}}
\newcommand{\BuiltInTok}[1]{#1}
\newcommand{\ExtensionTok}[1]{#1}
\newcommand{\PreprocessorTok}[1]{\textcolor[rgb]{0.56,0.35,0.01}{\textit{#1}}}
\newcommand{\AttributeTok}[1]{\textcolor[rgb]{0.77,0.63,0.00}{#1}}
\newcommand{\RegionMarkerTok}[1]{#1}
\newcommand{\InformationTok}[1]{\textcolor[rgb]{0.56,0.35,0.01}{\textbf{\textit{#1}}}}
\newcommand{\WarningTok}[1]{\textcolor[rgb]{0.56,0.35,0.01}{\textbf{\textit{#1}}}}
\newcommand{\AlertTok}[1]{\textcolor[rgb]{0.94,0.16,0.16}{#1}}
\newcommand{\ErrorTok}[1]{\textcolor[rgb]{0.64,0.00,0.00}{\textbf{#1}}}
\newcommand{\NormalTok}[1]{#1}
\usepackage{graphicx,grffile}
\makeatletter
\def\maxwidth{\ifdim\Gin@nat@width>\linewidth\linewidth\else\Gin@nat@width\fi}
\def\maxheight{\ifdim\Gin@nat@height>\textheight\textheight\else\Gin@nat@height\fi}
\makeatother
% Scale images if necessary, so that they will not overflow the page
% margins by default, and it is still possible to overwrite the defaults
% using explicit options in \includegraphics[width, height, ...]{}
\setkeys{Gin}{width=\maxwidth,height=\maxheight,keepaspectratio}
\IfFileExists{parskip.sty}{%
\usepackage{parskip}
}{% else
\setlength{\parindent}{0pt}
\setlength{\parskip}{6pt plus 2pt minus 1pt}
}
\setlength{\emergencystretch}{3em}  % prevent overfull lines
\providecommand{\tightlist}{%
  \setlength{\itemsep}{0pt}\setlength{\parskip}{0pt}}
\setcounter{secnumdepth}{0}
% Redefines (sub)paragraphs to behave more like sections
\ifx\paragraph\undefined\else
\let\oldparagraph\paragraph
\renewcommand{\paragraph}[1]{\oldparagraph{#1}\mbox{}}
\fi
\ifx\subparagraph\undefined\else
\let\oldsubparagraph\subparagraph
\renewcommand{\subparagraph}[1]{\oldsubparagraph{#1}\mbox{}}
\fi

%%% Use protect on footnotes to avoid problems with footnotes in titles
\let\rmarkdownfootnote\footnote%
\def\footnote{\protect\rmarkdownfootnote}

%%% Change title format to be more compact
\usepackage{titling}

% Create subtitle command for use in maketitle
\newcommand{\subtitle}[1]{
  \posttitle{
    \begin{center}\large#1\end{center}
    }
}

\setlength{\droptitle}{-2em}

  \title{STAT 331 Final Report}
    \pretitle{\vspace{\droptitle}\centering\huge}
  \posttitle{\par}
    \author{Nitisha Agarwal and Maya Perelman}
    \preauthor{\centering\large\emph}
  \postauthor{\par}
      \predate{\centering\large\emph}
  \postdate{\par}
    \date{12/1/2018}


\begin{document}
\maketitle

\subsection{I. Summary}\label{i.-summary}

The goal of this report is to explore the determinants of healthy male
birth weight in a sample population collected from a 1960s CHDS survey
in San Francisco. The dataset contains 1236 observations of healthy male
single-fetus births over 18 variates. Multiple linear regression allowed
connections to be made regarding the independent variables’ effects on
the dependent variable of healthy birth weight.

Unsurprisingly, the dataset had missing values and proper treatment of
these NAs was carried out through a variety of methods, including MICE.
The complete datasets obtained from imputation led to the development of
about 15 models. In order to select the best candidate models,
diagnostics such as cross validation and residual analysis were
employed.

In the end, the selected model was a modified version of the one
produced by automatic stepwise selection. During initial model
diagnostics, it was observed that this model had the lowest residual
standard errors, AIC, PRESS statistic scores, and the lowest rPMSE
obtained through cross validation. It was confirmed that the stepwise
model did not violate any of the assumptions of linear regression. To
obtain the final model, this stepwise model was tweaked to add and
remove interactions as deemed necessary.

\subsection{II. Model Selection}\label{ii.-model-selection}

\subsubsection{Pre-Fitting Data
Diagnostics}\label{pre-fitting-data-diagnostics}

In order to do some initial data exploration, we took a look at pairs
plots, summary of the data set, and dependant variable distribution. The
noteworthy points are presented below.

\includegraphics{Final_WriteUp_files/figure-latex/unnamed-chunk-1-1.pdf}
\(\textbf{Figure 1}\): Pairs plot across select variables.

We can make several observations from Figure 1. The values for gestation
are very concentrated between 250-300 which is expected for normal human
gestation period. From the data collected for the marital covariate, we
see a distinct line at marital = 1 which tells us that a large majority
of moms are married. We also see that there are 2 observations encoded
as marital = 0, despite 0 not being an option for marital status. This
is a consideration that we addressed when re-coding our data. Moreover,
we notice that high values of parity (\textgreater{}4) are uncommon,
which is unsurprising.

Our dependent variable of healthy male birth weight should have a normal
distribution, as the average weight of a healthy baby is known. Let's
verify this claim with a histogram.

\includegraphics{Final_WriteUp_files/figure-latex/unnamed-chunk-2-1.pdf}

\(\textbf{Figure 2}\): Distribution of dependent variable.

As predicted, the distribution of birth weight is in fact normally
distributed and well approximated by the Gaussian distribution as shown
in Figure 2.

\subsubsection{Treatment of NA's}\label{treatment-of-nas}

We run an NA count on our \texttt{chdbirths} data set to determine how
to proceed with treatment of missing data.

\begin{verbatim}
##        wt gestation    parity      meth      mage       med       mht 
##         0        13         0         1         2         1        22 
##       mwt      feth      fage       fed       fht       fwt   marital 
##        36        31         7        13       492       499         0 
##    income     smoke      time    number 
##       124        10        10        21
\end{verbatim}

\(\;\)\\
Immediately, we notice that father weight and height are missing almost
40\% of values, so we will have to discard these variables from our
analysis. After doing so, we consider a visual map of the distribution
of NAs (yellow) among our data set using the \texttt{Amelia} package in
R. Fun fact: the Amelia package in R is used for treatment of missing
data, and named after Amelia Earhart.

\(\;\)

\begin{verbatim}
## Error in axis(2, lwd = 0, labels = y.labels, las = 1, at = y.at, cex.axis = y.cex): no locations are finite
\end{verbatim}

\includegraphics{Final_WriteUp_files/figure-latex/unnamed-chunk-5-1.pdf}

\(\textbf{Figure 3}\): Missingness map for NAs by covariate.

As we can see there are a significant amount of NA's in the income
covariate, around 10\% of the observations, and we will want to impute
this in order to not lose a valuable predictor.

We will use the R \texttt{mice} package to do multiple imputation on our
missing values in the remaining continous variables (refer to Appendix
A). In particular, the imputation on income is important since around
10\% of its values are missing, and removing these could bias our
analysis. For the remaining NAs in categorical variables we simply
remove those observations. As a result, we only remove 55 observations,
which is around 6\% of our data set.

\begin{center}\includegraphics{Final_WriteUp_files/figure-latex/unnamed-chunk-7-1} \end{center}

\(\textbf{Figure 4}\): Plot to display difference between imputed values
(pink) and observed (blue).

These density plots in Figure 4 show the distribution of the data of
existing observations (blue) compared to the distribution of the imputed
values (pink) for each iteration of the MICE algorithm. As we can see,
the imputation of father age varies greatly because there are only 7
missing values, creating variability. The other distributions of imputed
values seem to be consistent with the existing data.

\subsubsection{Automatic Model
Selection}\label{automatic-model-selection}

We will now run the 3 automated model selection algorithms over each of
our 5 imputed data sets. This may generate up to 15 different models
(some may be the same formula). Refer to \emph{Appendix B} for the code
used to run automated model selection. In order to narrow down to the
best candidate models, we considered 3 main metrics of model quality,
summarized in the table below.

\subsubsection{Manual Model Selection}\label{manual-model-selection}

\begin{verbatim}
##             Ms1       Ms2       Ms3       Ms4       Ms5
## AIC     9687.35   9685.75   9692.72   9678.55   9697.28
## RSE       15.22     15.23     15.27     15.15     15.31
## PRESS 278739.89 278175.75 279837.42 276374.66 280983.31
\end{verbatim}

\(\textbf{Table 1}\): Summary statistics for stepwise selection models
over 5 imputed data sets.

Based off of these metrics, the model produced by Stepwise selection on
the 4th imputed data set had the lowest AIC, RSE, and PRESS, which
indicates that it's a better model fit. We will proceed with this one to
our cross-validation step. Next, here is the summary table for our
\emph{Forward selection} models.

\begin{verbatim}
##             Mf1       Mf2       Mf3       Mf4       Mf5
## AIC     9710.82   9707.19   9717.34   9700.48   9715.35
## RSE       15.40     15.39     15.45     15.37     15.44
## PRESS 284608.71 283617.99 285765.65 281067.63 285123.72
\end{verbatim}

\(\textbf{Table 2}\): Summary statistics for forward selection models
over 5 imputed data sets.

We can observe that the 4th model has the lowest RSE, AIC, and PRESS
statistics scores. As such, we will proceed with 4th forward model to
the cross-validation step. Next, we have the summary table for our
\emph{Backward selection} models.

\begin{verbatim}
##          Mb1     Mb2     Mb3     Mb4     Mb5
## AIC   9692.2 9689.97 9781.62 9684.33 9698.15
## RSE     15.2   15.18   15.36   15.13   15.23
## PRESS    Inf     Inf     Inf     Inf     Inf
\end{verbatim}

\(\textbf{Table 3}\): Summary statistics for backward elimination models
over 5 imputed data sets.

Based off of these metrics, the model produced by backward elimination
on the 4th imputed data set had the lowest AIC, RSE, and PRESS, which
indicates that it's a better model fit. However, the model call is very
convoluted and includes a multitude of interactions that are hard to
justify logically.As such, we proceed with our forward and stepwise
models to cross-validation.

\subsubsection{Cross-Validation}\label{cross-validation}

We will run cross-validation on the 2 candidate models \texttt{Ms4}, and
\texttt{Mf4}. The resulting boxplot of RPMSE and histogram of logLambda
are shown below.

\includegraphics{Final_WriteUp_files/figure-latex/unnamed-chunk-14-1.pdf}
\(\textbf{Figure 5}\): Boxplot (left) and histogram(right) of results of
cross-validation.

From the boxplot above we see that the stepwise model is comparitively
the best one. The mean red line on the logLambda histogram is negative,
indicating a preference for the stepwise model. Likewise, form hte
boxplot we observe a preference for the stepwise model as well.

\subsection{III. Model Diagnostics}\label{iii.-model-diagnostics}

We now compare and contrast the residuals of each model to make our
final choice for the best candidate between stepwise and forward
selection.

First, let's take a look at each model's residual plots.

\includegraphics{Final_WriteUp_files/figure-latex/unnamed-chunk-15-1.pdf}
\(\textbf{Figure 6}\): Residual plots for stepwise model.

From these 4 diagnostic plots we can observe that there is no clear
violation of the 3 assumptions of linear regression for the stepwise
model. The residuals vs.~fitted plot depicts an overall linear
conditional mean, with randomly scattered residuals about the x=0
horizontal line. We observe a slight deviation of the red mean (of
residual values at each fitted value) line from x=0 for higher fitted
values, but since so few observations fall past a fitted value of 150,
this is not too concerning.

In the QQ-plot we notice a slight deviation at the tails, but overall
the residuals of the stepwise model appear Normally distributed. The
Scale-Location plot shows us that the assumption of equal variance
(homoscedasticity) holds for this model, with a slight (but not
significant) deviation for higher fitted values. Finally, the Residuals
vs.~Leverage plot shows us that there are several points with high
leverage in the stepwise model, but none of them have a concerning
influence (if they were outside the dotted red lines depicting Cook's
distance).

We will now consider these same plots for the \emph{forward selection
model}.

\includegraphics{Final_WriteUp_files/figure-latex/unnamed-chunk-16-1.pdf}
\(\textbf{Figure 7}\): Residual plots for forward model.

Overall, the residual diagnostic plots look very similar to those for
the stepwise model. We will highlight a couple of differences.

First, the red mean line in the Residuals vs.~Fitted appears a little
bit straighter, indicating that the linear conditional mean assumption
holds a bit better for the forward model. The QQ-plot and Scale-Location
plot have little to no differences to those for the stepwise model. The
Residuals vs.~Leverage plot once again shows only points with high
leverage, and no particularly influential points. As in an ideal
scenario, we can not even see the Cook's distance lines in the graph.

The next thing we can take a look at are histograms of each model's
studentized residuals.

\includegraphics{Final_WriteUp_files/figure-latex/unnamed-chunk-17-1.pdf}
\(\textbf{Figure 8}\): Histogram of studentized residuals for both
models.

Once again, we see little to no differences between each model's
residuals, now studentized and standardized. Both distributions are very
slightly left skewed, supporting our previous observation of the
residuals being somewhat negatively biased. Both also have slight peaks
at the mean of 0 but are still well approximated by a Gaussian
distribution. So we can say that both models do not violate the 6th
assumption of regression, which claims iid Normal errors.

Now let's take a look at the influence vs.~leverage plot for both of our
models. Below, we plot the stepwise model in red and forwards in blue.

\begin{center}\includegraphics{Final_WriteUp_files/figure-latex/unnamed-chunk-18-1} \end{center}

\(\textbf{Figure 9}\): Influence vs.~Leverage plot for both models.

We can see that the stepwise model and the forwards model have a
majority of the points in the same area of the graph, with a few slight
exceptions. The stepwise model contains a few distinct high leverage
points (labelled on the right) and a few high influence points (labelled
near the top). Let's take a look at which observations these are to try
and deduce why they may have high leverage/influence.

\begin{verbatim}
##       wt gestation parity meth mage marital income smoke
## 1100 174       284      7   AA   39   other      1 never
\end{verbatim}

The highest influence point is an observation with one of the highest
birth weights, but a relatively normal gestational period. We take a
look at the other observations with equal or higher birth weight.

\begin{verbatim}
##       wt gestation parity      meth mage marital income smoke
## 557  174       281      3 Caucasian   37 married      7 never
## 633  176       293      0 Caucasian   19 married     NA never
## 748  174       288      2 Caucasian   25 married      3 never
## 1100 174       284      7        AA   39   other      1 never
\end{verbatim}

We notice that other babies born with such a high birth weight were all
born to Caucasian, married parents, unlike the highest influence point.
Moreover, our observation had an income level of 1, which is the lowest
of this group. These are the likely causes for the high influence of
this observation.

Now looking at the highest leverage point.

\begin{verbatim}
##      wt gestation parity meth mage med mht mwt feth fage fed marital
## 246 116       148      7   AA   28 mid  66 135   AA   36 mid   other
##     income smoke time number
## 246      2 never    0      0
\end{verbatim}

Looking at the highest leverage point, it is noticeable that observation
246 has a normal birth weight (mean birth weight is 119 ounces), and yet
a very low gestational period (of about 21 weeks). From research on
typical fetus weight gain during pregnancy, a fetus at 21 weeks is
typically \emph{12 ounces}, while this observation has a weight of
\emph{116 ounces}. This strikes us as a possible input error or improper
identification of conception date on the part of the parents, that may
have biased our automatic model selection.

After looking at various model diagnostic measures, we have noticed very
little differences between our two candidate models in terms of how well
the linear regression assumptions hold. From our earlier analysis using
cross validation, AIC, and PRESS statistics, the stepwise model was
superior to our forward model on all 3 counts. As such, as our final
model, we will select the one produced by \emph{stepwise selection}.

As discussed above however, our stepwise model may have been influenced
by the high leverage point. After re-running automatic stepwise
selection on all 5 imputed data sets \emph{with observation 246
excluded}, we noticed that our optimal stepwise model call changed
(refer to Appendix for details). The new stepwise model has a simpler
formula with less terms, and most notably, now includes the
\texttt{time:number} interaction. From our research, the metric
``pack-years'' which counts how much and for how long an individual
smoked is quite influential on birth weight, and we believe that the
\texttt{time:number} interaction is the best proxy for this measure.

Running cross-validation across all 5 \emph{complete} imputed data sets
for the old and new stepwise selection models revealead that they had no
significant difference in predictive power. As such, we will use the new
stepwise model as our final model since it retains the predicitve power
of the first one while being a simpler model. Refer to our Discussion
section below for a more detailed explanation on why we felt it was
justified to exclude observation 246.

As a final step, we pool the coefficients from the five imputations for
our stepwise model, and display the results including the final model's
call.

\begin{verbatim}
## lm(formula = wt ~ gestation + parity + meth + mage + mht + mwt + 
##     marital + income + smoke + time + number + gestation:income + 
##     gestation:number + mwt:income + gestation:mage + parity:marital + 
##     mht:marital + gestation:mwt + gestation:mht + time:number, 
##     data = compldata)
\end{verbatim}

\begin{verbatim}
##                     estimate std.error statistic       df p.value
## (Intercept)          305.793   221.794     1.379  912.765   0.168
## gestation             -0.925     0.796    -1.162  885.093   0.246
## parity                 0.926     0.294     3.147  903.663   0.002
## methAsian              4.248     2.754     1.543 1094.444   0.123
## methCaucasian          8.981     1.290     6.963  923.520   0.000
## methMexican           12.095     2.717     4.452 1122.733   0.000
## methOther              6.108     3.450     1.770 1129.902   0.077
## mage                  -3.818     1.627    -2.347  597.076   0.019
## mht                   -4.394     3.584    -1.226  940.448   0.220
## mwt                    0.991     0.423     2.345  864.446   0.019
## maritalother        -214.341    96.762    -2.215 1118.249   0.027
## income                -9.249     4.580    -2.019   79.415   0.044
## smokesmokesnow       -14.252     9.115    -1.564 1135.724   0.118
## smokeuntilpreg        -4.110     8.440    -0.487 1135.886   0.626
## smokeusedto           -0.046     6.604    -0.007 1137.381   0.994
## time                   1.221     1.163     1.050 1134.236   0.294
## number               -18.980     4.333    -4.380  968.910   0.000
## gestation:income       0.024     0.016     1.513   67.054   0.130
## gestation:number       0.052     0.014     3.825  972.111   0.000
## mwt:income             0.018     0.011     1.655   79.187   0.098
## gestation:mage         0.013     0.006     2.329  592.255   0.020
## parity:maritalother    4.066     1.413     2.878 1102.028   0.004
## mht:maritalother       3.183     1.503     2.118 1119.781   0.034
## gestation:mwt         -0.003     0.002    -2.313  907.024   0.021
## gestation:mht          0.019     0.013     1.511  924.477   0.131
## time:number            0.423     0.261     1.623 1131.287   0.105
\end{verbatim}

\subsubsection{Errors and Trouble
Encountered}\label{errors-and-trouble-encountered}

During our data analysis, we learned a lot about strategies to choose a
well fitting model for a given sample population. In the end we retained
a model that made sense to us, but to get there we did first run into a
few hurdles. For example, when working with models with quite a few
terms, we received an error of “Prediction from a rank-deficient fit
may be misleading” which our research explained by claiming that as
you add terms to a formula, regression will consider having too many
predictors as a problem. To deal with this, we tried taking out a few
interactions that weren’t intuitive and re-running cross-validation to
see if the removal helped. Another issue we encountered was that
interactions between variables such as parent's education and ethnicity
which produced a lot of NA’s so we decided to restrict our models from
including such interactions.

\subsection{IV. Discussion}\label{iv.-discussion}

\begin{enumerate}
\def\labelenumi{\arabic{enumi}.}
\item
  From our analysis, we have explored the relation of the 17 independent
  variables given in our data set to healthy singleton male births in
  San Francisco in the 1960's. From our models, we have noticed that the
  most significant predictors for our given population included both
  physical factors and socioeconomic factors. In terms of physical
  attributes, length of gestational period was particularly predictive,
  as well as mother's height/weight and parity (whether the child was
  first born or not). In addition, the smoking status of the mother
  (whether/how much they smoked during pregnancy) as well as ethnicity
  (Caucasian tended to have higher birth weights) were noticed to be
  predictive. In terms of socioeconomic factors, income was a predictive
  variable in our models, with higher-income units typically having
  higher birth weights. However, we cannot make any conclusion about
  marital status based off of our model since 98\% of our observations
  were married couples.
\item
  Based off of our analysis, the biggest behavioural changes that could
  be made in order to avoid low birth weight would be to quit smoking as
  soon as possible, whether that be before pregnancy or as soon as
  pregnancy is known. As well, potentially ensuring that the mother has
  a healthy weight, and in particular is not significantly underweight,
  could help lower the risk of a low birth weight. As well, although not
  in our final model, from our research we haven noted that studies have
  found a strong association between illiteracy and low birth weight.
  This is another factor that could potentially be in control of the
  mother, depending on her socioeconomic resources.
\item
  Yes, in our final model we retained coefficients with high p-values
  (\textgreater{}0.05). This was because the basis for adding/removing
  terms in the automatic stepwise selection algorithm was carried out
  based on AIC score, and not on p-value. As such, the algorithm
  optimized a different measure of statiscial significance, and it is
  still appropriate to keep these coefficients in our final model.
\item
  Yes, as previously discussed when analyzing leverage above,
  observation 246 had an almost impossibly high birth weight given that
  the gestational period was around 21 weeks. This observation be
  appropriate to remove because even it if wasn't caused by incorrect
  measurement, is statistically very improbably to be repeated, and we
  wouldn't necessarily want a model to be influenced by it. As well, we
  noticed that there were quite a few observations with unusually high
  gestational periods. In particular, observations 11 and 1104 had
  gestational periods over 350 days, which is almost a year in length.
  Although removing these observations would likely not be appropriate,
  this makes us question exactly how gestational period was defined, and
  how likely the mother was to provide an accurate conception date. This
  might be an important consideration for any future studies conducted
  on this topic.
\item
  As discussed in the model selection above, all of the assumptions of
  linear regression can be said to hold for our final model. Possible
  deficiences of our model is that it was trained on a very specific
  subpopulation. In the model diagnostics section above, we saw that
  both models failed to providence any evidence of a violation of
  regression assumptions. A possible deficiency of our model is that it
  was trained on a very specific subpopulation from 1960s San Francisco.
  This subpopulation has certain characteristics that may not be seen in
  more recently collected data or in varying geographic locations. Due
  to this, even our cross validation tests may not have been entirely
  representative of out-of-sample prediction. This may have biased us
  towards selecting a final model with a larger amount of terms in the
  formula.
\end{enumerate}

\subsection{Appendix for R Code}\label{appendix-for-r-code}

\subsubsection{A. Pre-Fitting}\label{a.-pre-fitting}

\begin{Shaded}
\begin{Highlighting}[]
\CommentTok{# Creating pairs plots, adjusting which variables we want in it}
\KeywordTok{pairs}\NormalTok{(chdbirths[,}\KeywordTok{c}\NormalTok{(}\DecValTok{1}\NormalTok{,}\DecValTok{2}\NormalTok{,}\DecValTok{3}\NormalTok{,}\DecValTok{14}\NormalTok{,}\DecValTok{8}\NormalTok{)],}\DataTypeTok{pch=}\DecValTok{16}\NormalTok{, }\DataTypeTok{cex=}\NormalTok{.}\DecValTok{3}\NormalTok{, }\DataTypeTok{col=}\KeywordTok{adjustcolor}\NormalTok{(}\StringTok{"black"}\NormalTok{,}\FloatTok{0.5}\NormalTok{))}

\CommentTok{# Create histogram of birthweight to view distribution}
\KeywordTok{hist}\NormalTok{(chdbirths}\OperatorTok{$}\NormalTok{wt,}\DataTypeTok{freq=}\OtherTok{FALSE}\NormalTok{, }\DataTypeTok{breaks=}\StringTok{"FD"}\NormalTok{, }\DataTypeTok{col=}\StringTok{"gray90"}\NormalTok{,}
     \DataTypeTok{xlab =} \StringTok{"Healthy Male Birth Weight"}\NormalTok{, }\DataTypeTok{main =} \StringTok{"Histogram of Birth Weight"}\NormalTok{)}
\CommentTok{# Overlay Gaussian approximation}
\KeywordTok{curve}\NormalTok{(}\KeywordTok{dnorm}\NormalTok{(x,}\DataTypeTok{mean=}\KeywordTok{mean}\NormalTok{(chdbirths}\OperatorTok{$}\NormalTok{wt),}\DataTypeTok{sd=}\KeywordTok{sd}\NormalTok{(chdbirths}\OperatorTok{$}\NormalTok{wt)),}\DataTypeTok{col=}\StringTok{"red"}\NormalTok{,}\DataTypeTok{add=}\OtherTok{TRUE}\NormalTok{)}


\NormalTok{## Re-coding the data that doesn't make sense ordinally into categorical variables}

\CommentTok{# Mother and father ethnicity}
\NormalTok{births}\OperatorTok{$}\NormalTok{meth[chdbirths}\OperatorTok{$}\NormalTok{meth }\OperatorTok\StringTok{ }\KeywordTok{c}\NormalTok{(}\DecValTok{0}\NormalTok{,}\DecValTok{1}\NormalTok{,}\DecValTok{2}\NormalTok{,}\DecValTok{3}\NormalTok{,}\DecValTok{4}\NormalTok{,}\DecValTok{5}\NormalTok{)] <-}\StringTok{ "Caucasian"}
\NormalTok{births}\OperatorTok{$}\NormalTok{meth[chdbirths}\OperatorTok{$}\NormalTok{meth }\OperatorTok{==}\StringTok{ }\DecValTok{6}\NormalTok{] <-}\StringTok{ "Mexican"}
\NormalTok{births}\OperatorTok{$}\NormalTok{meth[chdbirths}\OperatorTok{$}\NormalTok{meth }\OperatorTok{==}\StringTok{ }\DecValTok{7}\NormalTok{] <-}\StringTok{ "AA"}
\NormalTok{births}\OperatorTok{$}\NormalTok{meth[chdbirths}\OperatorTok{$}\NormalTok{meth }\OperatorTok{==}\StringTok{ }\DecValTok{8}\NormalTok{] <-}\StringTok{ "Asian"}
\NormalTok{births}\OperatorTok{$}\NormalTok{meth[chdbirths}\OperatorTok{$}\NormalTok{meth }\OperatorTok\StringTok{ }\KeywordTok{c}\NormalTok{(}\DecValTok{9}\NormalTok{,}\DecValTok{10}\NormalTok{)] <-}\StringTok{ "Other"} \CommentTok{# grouping in Mixed }

\NormalTok{births}\OperatorTok{$}\NormalTok{feth[chdbirths}\OperatorTok{$}\NormalTok{feth }\OperatorTok\StringTok{ }\KeywordTok{c}\NormalTok{(}\DecValTok{0}\NormalTok{,}\DecValTok{1}\NormalTok{,}\DecValTok{2}\NormalTok{,}\DecValTok{3}\NormalTok{,}\DecValTok{4}\NormalTok{,}\DecValTok{5}\NormalTok{)] <-}\StringTok{ "Caucasian"}
\NormalTok{births}\OperatorTok{$}\NormalTok{feth[chdbirths}\OperatorTok{$}\NormalTok{feth }\OperatorTok{==}\StringTok{ }\DecValTok{6}\NormalTok{] <-}\StringTok{ "Mexican"}
\NormalTok{births}\OperatorTok{$}\NormalTok{feth[chdbirths}\OperatorTok{$}\NormalTok{feth }\OperatorTok{==}\StringTok{ }\DecValTok{7}\NormalTok{] <-}\StringTok{ "AA"}
\NormalTok{births}\OperatorTok{$}\NormalTok{feth[chdbirths}\OperatorTok{$}\NormalTok{feth }\OperatorTok{==}\StringTok{ }\DecValTok{8}\NormalTok{] <-}\StringTok{ "Asian"}
\NormalTok{births}\OperatorTok{$}\NormalTok{feth[chdbirths}\OperatorTok{$}\NormalTok{feth }\OperatorTok\StringTok{ }\KeywordTok{c}\NormalTok{(}\DecValTok{9}\NormalTok{,}\DecValTok{10}\NormalTok{)] <-}\StringTok{ "Other"} \CommentTok{# grouping in Mixed }

\CommentTok{# Mother and father education}
\NormalTok{births}\OperatorTok{$}\NormalTok{med[chdbirths}\OperatorTok{$}\NormalTok{med }\OperatorTok{==}\StringTok{ }\DecValTok{0}\NormalTok{] <-}\StringTok{ "elem"}
\NormalTok{births}\OperatorTok{$}\NormalTok{med[chdbirths}\OperatorTok{$}\NormalTok{med }\OperatorTok{==}\StringTok{ }\DecValTok{1}\NormalTok{] <-}\StringTok{ "mid"}
\NormalTok{births}\OperatorTok{$}\NormalTok{med[chdbirths}\OperatorTok{$}\NormalTok{med }\OperatorTok{==}\StringTok{ }\DecValTok{2}\NormalTok{] <-}\StringTok{ "hs"}
\NormalTok{births}\OperatorTok{$}\NormalTok{med[chdbirths}\OperatorTok{$}\NormalTok{med }\OperatorTok{==}\StringTok{ }\DecValTok{3}\NormalTok{] <-}\StringTok{ "hs+trade"}
\NormalTok{births}\OperatorTok{$}\NormalTok{med[chdbirths}\OperatorTok{$}\NormalTok{med }\OperatorTok{==}\StringTok{ }\DecValTok{4}\NormalTok{] <-}\StringTok{ "hs+some+col"}
\NormalTok{births}\OperatorTok{$}\NormalTok{med[chdbirths}\OperatorTok{$}\NormalTok{med }\OperatorTok{==}\StringTok{ }\DecValTok{5}\NormalTok{] <-}\StringTok{ "col"}
\NormalTok{births}\OperatorTok{$}\NormalTok{med[chdbirths}\OperatorTok{$}\NormalTok{med }\OperatorTok{==}\StringTok{ }\DecValTok{6}\NormalTok{] <-}\StringTok{ "trade"}
\NormalTok{births}\OperatorTok{$}\NormalTok{med[chdbirths}\OperatorTok{$}\NormalTok{med }\OperatorTok{==}\StringTok{ }\DecValTok{7}\NormalTok{] <-}\StringTok{ "hs+unclear"}

\NormalTok{births}\OperatorTok{$}\NormalTok{fed[chdbirths}\OperatorTok{$}\NormalTok{fed }\OperatorTok{==}\StringTok{ }\DecValTok{0}\NormalTok{] <-}\StringTok{ "elem"}
\NormalTok{births}\OperatorTok{$}\NormalTok{fed[chdbirths}\OperatorTok{$}\NormalTok{fed }\OperatorTok{==}\StringTok{ }\DecValTok{1}\NormalTok{] <-}\StringTok{ "mid"}
\NormalTok{births}\OperatorTok{$}\NormalTok{fed[chdbirths}\OperatorTok{$}\NormalTok{fed }\OperatorTok{==}\StringTok{ }\DecValTok{2}\NormalTok{] <-}\StringTok{ "hs"}
\NormalTok{births}\OperatorTok{$}\NormalTok{fed[chdbirths}\OperatorTok{$}\NormalTok{fed }\OperatorTok{==}\StringTok{ }\DecValTok{3}\NormalTok{] <-}\StringTok{ "hs+trade"}
\NormalTok{births}\OperatorTok{$}\NormalTok{fed[chdbirths}\OperatorTok{$}\NormalTok{fed }\OperatorTok{==}\StringTok{ }\DecValTok{4}\NormalTok{] <-}\StringTok{ "hs+some+col"}
\NormalTok{births}\OperatorTok{$}\NormalTok{fed[chdbirths}\OperatorTok{$}\NormalTok{fed }\OperatorTok{==}\StringTok{ }\DecValTok{5}\NormalTok{] <-}\StringTok{ "col"}
\NormalTok{births}\OperatorTok{$}\NormalTok{fed[chdbirths}\OperatorTok{$}\NormalTok{fed }\OperatorTok{==}\StringTok{ }\DecValTok{6}\NormalTok{] <-}\StringTok{ "trade"}
\NormalTok{births}\OperatorTok{$}\NormalTok{fed[chdbirths}\OperatorTok{$}\NormalTok{fed }\OperatorTok{==}\StringTok{ }\DecValTok{7}\NormalTok{] <-}\StringTok{ "hs+unclear"}

\CommentTok{# Marital Status}
\NormalTok{births}\OperatorTok{$}\NormalTok{marital[chdbirths}\OperatorTok{$}\NormalTok{marital }\OperatorTok{==}\StringTok{ }\DecValTok{1}\NormalTok{] <-}\StringTok{ "married"}
\CommentTok{# Grouping all other types because vast majority of obs. are married}
\NormalTok{births}\OperatorTok{$}\NormalTok{marital[chdbirths}\OperatorTok{$}\NormalTok{marital }\OperatorTok\StringTok{ }\KeywordTok{c}\NormalTok{(}\DecValTok{0}\NormalTok{,}\DecValTok{2}\NormalTok{,}\DecValTok{3}\NormalTok{,}\DecValTok{4}\NormalTok{,}\DecValTok{5}\NormalTok{)] <-}\StringTok{ "other"}

\CommentTok{# Smoke}
\NormalTok{births}\OperatorTok{$}\NormalTok{smoke[chdbirths}\OperatorTok{$}\NormalTok{smoke }\OperatorTok{==}\StringTok{ }\DecValTok{0}\NormalTok{] <-}\StringTok{ "never"}
\NormalTok{births}\OperatorTok{$}\NormalTok{smoke[chdbirths}\OperatorTok{$}\NormalTok{smoke }\OperatorTok{==}\StringTok{ }\DecValTok{1}\NormalTok{] <-}\StringTok{ "smokesnow"}
\NormalTok{births}\OperatorTok{$}\NormalTok{smoke[chdbirths}\OperatorTok{$}\NormalTok{smoke }\OperatorTok{==}\StringTok{ }\DecValTok{2}\NormalTok{] <-}\StringTok{ "untilpreg"}
\NormalTok{births}\OperatorTok{$}\NormalTok{smoke[chdbirths}\OperatorTok{$}\NormalTok{smoke }\OperatorTok{==}\StringTok{ }\DecValTok{3}\NormalTok{] <-}\StringTok{ "usedto"}

\CommentTok{#Time, keep ordinal but change to more sensible order}
\NormalTok{births}\OperatorTok{$}\NormalTok{time[chdbirths}\OperatorTok{$}\NormalTok{time }\OperatorTok{==}\StringTok{ }\DecValTok{1}\NormalTok{] <-}\StringTok{ }\DecValTok{8}
\NormalTok{births}\OperatorTok{$}\NormalTok{time[chdbirths}\OperatorTok{$}\NormalTok{time }\OperatorTok{==}\StringTok{ }\DecValTok{2}\NormalTok{] <-}\StringTok{ }\DecValTok{7}
\NormalTok{births}\OperatorTok{$}\NormalTok{time[chdbirths}\OperatorTok{$}\NormalTok{time }\OperatorTok{==}\StringTok{ }\DecValTok{3}\NormalTok{] <-}\StringTok{ }\DecValTok{6}
\NormalTok{births}\OperatorTok{$}\NormalTok{time[chdbirths}\OperatorTok{$}\NormalTok{time }\OperatorTok{==}\StringTok{ }\DecValTok{4}\NormalTok{] <-}\StringTok{ }\DecValTok{5}
\NormalTok{births}\OperatorTok{$}\NormalTok{time[chdbirths}\OperatorTok{$}\NormalTok{time }\OperatorTok{==}\StringTok{ }\DecValTok{5}\NormalTok{] <-}\StringTok{ }\DecValTok{4}
\NormalTok{births}\OperatorTok{$}\NormalTok{time[chdbirths}\OperatorTok{$}\NormalTok{time }\OperatorTok{==}\StringTok{ }\DecValTok{6}\NormalTok{] <-}\StringTok{ }\DecValTok{3}
\NormalTok{births}\OperatorTok{$}\NormalTok{time[chdbirths}\OperatorTok{$}\NormalTok{time }\OperatorTok{==}\StringTok{ }\DecValTok{7}\NormalTok{] <-}\StringTok{ }\DecValTok{2}
\NormalTok{births}\OperatorTok{$}\NormalTok{time[chdbirths}\OperatorTok{$}\NormalTok{time }\OperatorTok{==}\StringTok{ }\DecValTok{8}\NormalTok{] <-}\StringTok{ }\DecValTok{1}
\NormalTok{births}\OperatorTok{$}\NormalTok{time[chdbirths}\OperatorTok{$}\NormalTok{time }\OperatorTok{==}\StringTok{ }\DecValTok{9}\NormalTok{] <-}\StringTok{ }\OtherTok{NA} \CommentTok{# only 5 observations, will be deleted}

\NormalTok{##}


\CommentTok{# Counting NA's in each variate}
\NormalTok{NAcount <-}\StringTok{ }\KeywordTok{sapply}\NormalTok{(chdbirths, }\ControlFlowTok{function}\NormalTok{(x) }\KeywordTok{sum}\NormalTok{(}\KeywordTok{is.na}\NormalTok{(x)))}

\CommentTok{# Dropping father height and weight since so many NA's }
\NormalTok{drops <-}\StringTok{ }\KeywordTok{c}\NormalTok{(}\StringTok{"fwt"}\NormalTok{,}\StringTok{"fht"}\NormalTok{)}
\NormalTok{births <-}\StringTok{ }\NormalTok{births[,}\OperatorTok{!}\NormalTok{(}\KeywordTok{names}\NormalTok{(births) }\OperatorTok\StringTok{ }\NormalTok{drops)]}

\CommentTok{# Using Amelia R package, display a graph of proportion of missing vals}
\KeywordTok{missmap}\NormalTok{(births, }\DataTypeTok{col=}\KeywordTok{c}\NormalTok{(}\StringTok{"yellow"}\NormalTok{,}\StringTok{"midnightblue"}\NormalTok{),}\DataTypeTok{y.labels=}\StringTok{""}\NormalTok{,}\DataTypeTok{y.at=}\StringTok{""}\NormalTok{)}

\CommentTok{# Displaying differences between observations and imputed data}
\KeywordTok{densityplot}\NormalTok{(imp_all) }
\end{Highlighting}
\end{Shaded}

The following code was used to run the mice algorithm and create 5
imputed data sets.

\begin{Shaded}
\begin{Highlighting}[]
\CommentTok{#removing observations with NAs in variables that aren't income}
\NormalTok{birthsCompl <-}\StringTok{ }\NormalTok{births[}\KeywordTok{complete.cases}\NormalTok{(births}\OperatorTok{$}\NormalTok{marital, births}\OperatorTok{$}\NormalTok{meth,}
\NormalTok{                                     births}\OperatorTok{$}\NormalTok{med,births}\OperatorTok{$}\NormalTok{feth,}
\NormalTok{                                     births}\OperatorTok{$}\NormalTok{fed, births}\OperatorTok{$}\NormalTok{smoke, births}\OperatorTok{$}\NormalTok{time, births}\OperatorTok{$}\NormalTok{number),]}
\CommentTok{# using the mice function to impute missing values over 5 iterations}
\KeywordTok{set.seed}\NormalTok{(}\DecValTok{300}\NormalTok{)}
\NormalTok{imp_all <-}\StringTok{ }\KeywordTok{mice}\NormalTok{(birthsCompl,}\DataTypeTok{m=}\DecValTok{5}\NormalTok{)}
\end{Highlighting}
\end{Shaded}

\subsubsection{B. Model Selection}\label{b.-model-selection}

This code pertains to our Automatic Model Selection in the report.

\begin{Shaded}
\begin{Highlighting}[]
\CommentTok{# Stepwise Model Selection}
\ControlFlowTok{for}\NormalTok{ (i }\ControlFlowTok{in} \DecValTok{1}\OperatorTok{:}\DecValTok{5}\NormalTok{)\{ }\CommentTok{# run through each imputation mice returns }
\NormalTok{  compldata <-}\StringTok{ }\KeywordTok{complete}\NormalTok{(imp_all,i) }\CommentTok{# Completing data set over each imputation}
\NormalTok{  M0 <-}\StringTok{ }\KeywordTok{lm}\NormalTok{(wt}\OperatorTok{~}\DecValTok{1}\NormalTok{,}\DataTypeTok{data=}\NormalTok{compldata) }\CommentTok{# minimal model: intercept only}
  
  \CommentTok{# all main effects and interactions}
\NormalTok{  ##  Removed feth, meth, fed, med interactions because they produced NA's}
\NormalTok{  Mmax <-}\StringTok{ }\KeywordTok{lm}\NormalTok{(wt }\OperatorTok{~}\StringTok{ }\NormalTok{(.}\OperatorTok{-}\NormalTok{feth }\OperatorTok{-}\StringTok{ }\NormalTok{meth }\OperatorTok{-}\StringTok{ }\NormalTok{fed }\OperatorTok{-}\StringTok{ }\NormalTok{med)}\OperatorTok{^}\DecValTok{2} \OperatorTok{+}\StringTok{ }\NormalTok{feth }\OperatorTok{+}\StringTok{ }\NormalTok{meth }\OperatorTok{+}\StringTok{ }\NormalTok{fed }\OperatorTok{+}\StringTok{ }\NormalTok{med,}\DataTypeTok{data=}\NormalTok{compldata)}
\NormalTok{  Mstart <-}\KeywordTok{lm}\NormalTok{(wt }\OperatorTok{~}\NormalTok{., }\DataTypeTok{data =}\NormalTok{ compldata) }\CommentTok{# starting model for stepwise}
\NormalTok{  Mstep <-}\StringTok{ }\KeywordTok{step}\NormalTok{(}\DataTypeTok{object=}\NormalTok{ Mstart,}
              \DataTypeTok{scope=}\KeywordTok{list}\NormalTok{(}\DataTypeTok{lower=}\NormalTok{M0, }\DataTypeTok{upper =}\NormalTok{ Mmax),}
              \DataTypeTok{direction =}\StringTok{"both"}\NormalTok{, }\CommentTok{# stepwise direction }
              \DataTypeTok{trace=}\OtherTok{FALSE}\NormalTok{)}
\NormalTok{  model <-}\StringTok{ }\KeywordTok{paste}\NormalTok{(}\StringTok{"Ms"}\NormalTok{,i, }\DataTypeTok{sep=}\StringTok{""}\NormalTok{) }\CommentTok{# dynamically name models with corresponding index}
  \KeywordTok{assign}\NormalTok{(model,Mstep)}
\NormalTok{\}}

\CommentTok{# Forward Model Selection}
\ControlFlowTok{for}\NormalTok{ (i }\ControlFlowTok{in} \DecValTok{1}\OperatorTok{:}\DecValTok{5}\NormalTok{)\{ }\CommentTok{# run through each imputation mice returns }
\NormalTok{  compldata <-}\StringTok{ }\KeywordTok{complete}\NormalTok{(imp_all,i) }\CommentTok{# Completing data set over each imputation}
\NormalTok{  Mfwd <-}\StringTok{ }\KeywordTok{step}\NormalTok{(}\DataTypeTok{object =}\NormalTok{ M0,}
              \DataTypeTok{scope=}\KeywordTok{list}\NormalTok{(}\DataTypeTok{lower=}\NormalTok{M0, }\DataTypeTok{upper =}\NormalTok{ Mmax),}
              \DataTypeTok{direction =}\StringTok{"forward"}\NormalTok{, }\CommentTok{# forward direction }
              \DataTypeTok{trace=}\OtherTok{FALSE}\NormalTok{)}
\NormalTok{  model <-}\StringTok{ }\KeywordTok{paste}\NormalTok{(}\StringTok{"Mf"}\NormalTok{,i, }\DataTypeTok{sep=}\StringTok{""}\NormalTok{) }\CommentTok{# dynamically name models with corresponding index}
  \KeywordTok{assign}\NormalTok{(model,Mfwd)}
\NormalTok{\}}

\CommentTok{# Backwards Model Selection}
\ControlFlowTok{for}\NormalTok{ (i }\ControlFlowTok{in} \DecValTok{1}\OperatorTok{:}\DecValTok{5}\NormalTok{)\{ }\CommentTok{# run through each imputation mice returns}
\NormalTok{  compldata <-}\StringTok{ }\KeywordTok{complete}\NormalTok{(imp_all,i) }\CommentTok{# Completing data set over each imputation}
\NormalTok{  Mback <-}\StringTok{ }\KeywordTok{step}\NormalTok{(}\DataTypeTok{object =}\NormalTok{ Mmax,}
              \DataTypeTok{scope=}\KeywordTok{list}\NormalTok{(}\DataTypeTok{lower=}\NormalTok{M0, }\DataTypeTok{upper =}\NormalTok{ Mmax),}
              \DataTypeTok{direction =}\StringTok{"backward"}\NormalTok{, }\CommentTok{# backwards direction }
              \DataTypeTok{trace=}\OtherTok{FALSE}\NormalTok{)}
\NormalTok{  model <-}\StringTok{ }\KeywordTok{paste}\NormalTok{(}\StringTok{"Mb"}\NormalTok{,i, }\DataTypeTok{sep=}\StringTok{""}\NormalTok{) }\CommentTok{# dynamically name models with corresponding index}
  \KeywordTok{assign}\NormalTok{(model,Mback)}
\NormalTok{\}}
\end{Highlighting}
\end{Shaded}

\subsubsection{C. Manual Model
Selection}\label{c.-manual-model-selection}

The following code pertains to our Manual Model Selection section of the
report, and is used to generate the 3 tables of diagnostic statistics on
the 15 models generated by automatic model selection.

\begin{Shaded}
\begin{Highlighting}[]
\CommentTok{# Calculate PRESS statistic for each of the 5 stepwise models, similar code for forward/backward}
\NormalTok{pressc1 <-}\StringTok{ }\KeywordTok{sum}\NormalTok{((}\KeywordTok{resid}\NormalTok{(Ms1)}\OperatorTok{/}\NormalTok{(}\DecValTok{1}\OperatorTok{-}\KeywordTok{hatvalues}\NormalTok{(Ms1)))}\OperatorTok{^}\DecValTok{2}\NormalTok{)}
\NormalTok{pressc2 <-}\StringTok{ }\KeywordTok{sum}\NormalTok{((}\KeywordTok{resid}\NormalTok{(Ms2)}\OperatorTok{/}\NormalTok{(}\DecValTok{1}\OperatorTok{-}\KeywordTok{hatvalues}\NormalTok{(Ms2)))}\OperatorTok{^}\DecValTok{2}\NormalTok{)}
\NormalTok{pressc3 <-}\StringTok{ }\KeywordTok{sum}\NormalTok{((}\KeywordTok{resid}\NormalTok{(Ms3)}\OperatorTok{/}\NormalTok{(}\DecValTok{1}\OperatorTok{-}\KeywordTok{hatvalues}\NormalTok{(Ms3)))}\OperatorTok{^}\DecValTok{2}\NormalTok{)}
\NormalTok{pressc4 <-}\StringTok{ }\KeywordTok{sum}\NormalTok{((}\KeywordTok{resid}\NormalTok{(Ms4)}\OperatorTok{/}\NormalTok{(}\DecValTok{1}\OperatorTok{-}\KeywordTok{hatvalues}\NormalTok{(Ms4)))}\OperatorTok{^}\DecValTok{2}\NormalTok{)}
\NormalTok{pressc5 <-}\StringTok{ }\KeywordTok{sum}\NormalTok{((}\KeywordTok{resid}\NormalTok{(Ms5)}\OperatorTok{/}\NormalTok{(}\DecValTok{1}\OperatorTok{-}\KeywordTok{hatvalues}\NormalTok{(Ms5)))}\OperatorTok{^}\DecValTok{2}\NormalTok{)}


\CommentTok{# Stepwise models summary of AIC, RSE, and PRESS statistic, similar code for forward/backward}
\NormalTok{step.matrix <-}\StringTok{ }\KeywordTok{matrix}\NormalTok{(}\DecValTok{0}\NormalTok{, }\DecValTok{3}\NormalTok{, }\DecValTok{5}\NormalTok{) }\CommentTok{# initial matrix }
\KeywordTok{rownames}\NormalTok{(step.matrix) <-}\StringTok{ }\KeywordTok{c}\NormalTok{(}\StringTok{"AIC"}\NormalTok{, }\StringTok{"RSE"}\NormalTok{, }\StringTok{"PRESS"}\NormalTok{)}
\KeywordTok{colnames}\NormalTok{(step.matrix) <-}\StringTok{ }\KeywordTok{c}\NormalTok{(}\StringTok{"Ms1"}\NormalTok{, }\StringTok{"Ms2"}\NormalTok{, }\StringTok{"Ms3"}\NormalTok{, }\StringTok{"Ms4"}\NormalTok{, }\StringTok{"Ms5"}\NormalTok{)}

\CommentTok{# Create vectors of AIC, RSE, PRESS for each model}
\NormalTok{aic.s <-}\StringTok{ }\KeywordTok{round}\NormalTok{(}\KeywordTok{c}\NormalTok{(}\KeywordTok{AIC}\NormalTok{(Ms1), }\KeywordTok{AIC}\NormalTok{(Ms2), }\KeywordTok{AIC}\NormalTok{(Ms3), }\KeywordTok{AIC}\NormalTok{(Ms4), }\KeywordTok{AIC}\NormalTok{(Ms5)),}\DecValTok{2}\NormalTok{)}
\NormalTok{rse.s <-}\StringTok{ }\KeywordTok{round}\NormalTok{(}\KeywordTok{c}\NormalTok{(}\KeywordTok{sigma}\NormalTok{(Ms1), }\KeywordTok{sigma}\NormalTok{(Ms2), }\KeywordTok{sigma}\NormalTok{(Ms3), }\KeywordTok{sigma}\NormalTok{(Ms4), }\KeywordTok{sigma}\NormalTok{(Ms5)),}\DecValTok{2}\NormalTok{)}
\NormalTok{press.s <-}\StringTok{ }\KeywordTok{round}\NormalTok{(}\KeywordTok{c}\NormalTok{(pressc1, pressc2, pressc3, pressc4, pressc5),}\DecValTok{2}\NormalTok{)}
\CommentTok{# Update matrix}
\NormalTok{step.matrix[}\DecValTok{1}\NormalTok{,] <-}\StringTok{ }\NormalTok{aic.s}
\NormalTok{step.matrix[}\DecValTok{2}\NormalTok{,] <-}\StringTok{ }\NormalTok{rse.s}
\NormalTok{step.matrix[}\DecValTok{3}\NormalTok{,] <-}\StringTok{ }\NormalTok{press.s}
\end{Highlighting}
\end{Shaded}

\subsubsection{D. Cross-Validation}\label{d.-cross-validation}

The following section of our code was used for our cross-validation of
the forward and stepwise models. It pertains to the cross-validation
section of our report.

\begin{Shaded}
\begin{Highlighting}[]
\CommentTok{# Fit stepwise model to imputations returned by mice }
\NormalTok{fit.step <-}\StringTok{ }\KeywordTok{with}\NormalTok{(}\DataTypeTok{data=}\NormalTok{imp_all, }\DataTypeTok{exp =} \KeywordTok{lm}\NormalTok{(}\DataTypeTok{formula =}\NormalTok{ wt }\OperatorTok{~}\StringTok{ }\NormalTok{gestation }\OperatorTok{+}\StringTok{ }\NormalTok{parity }\OperatorTok{+}\StringTok{ }\NormalTok{meth }\OperatorTok{+}\StringTok{ }\NormalTok{mage }\OperatorTok{+}\StringTok{ }
\StringTok{    }\NormalTok{mht }\OperatorTok{+}\StringTok{ }\NormalTok{mwt }\OperatorTok{+}\StringTok{ }\NormalTok{marital }\OperatorTok{+}\StringTok{ }\NormalTok{income }\OperatorTok{+}\StringTok{ }\NormalTok{smoke }\OperatorTok{+}\StringTok{ }\NormalTok{time }\OperatorTok{+}\StringTok{ }\NormalTok{number }\OperatorTok{+}\StringTok{ }\NormalTok{gestation}\OperatorTok{:}\NormalTok{income }\OperatorTok{+}\StringTok{ }
\StringTok{    }\NormalTok{gestation}\OperatorTok{:}\NormalTok{number }\OperatorTok{+}\StringTok{ }\NormalTok{parity}\OperatorTok{:}\NormalTok{marital }\OperatorTok{+}\StringTok{ }\NormalTok{mwt}\OperatorTok{:}\NormalTok{income }\OperatorTok{+}\StringTok{ }\NormalTok{mht}\OperatorTok{:}\NormalTok{marital }\OperatorTok{+}\StringTok{ }
\StringTok{    }\NormalTok{gestation}\OperatorTok{:}\NormalTok{mage }\OperatorTok{+}\StringTok{ }\NormalTok{gestation}\OperatorTok{:}\NormalTok{mwt }\OperatorTok{+}\StringTok{ }\NormalTok{gestation}\OperatorTok{:}\NormalTok{mht }\OperatorTok{+}\StringTok{ }\NormalTok{mage}\OperatorTok{:}\NormalTok{time }\OperatorTok{+}\StringTok{ }
\StringTok{    }\NormalTok{mage}\OperatorTok{:}\NormalTok{number }\OperatorTok{+}\StringTok{ }\NormalTok{smoke}\OperatorTok{:}\NormalTok{number))}


\CommentTok{# Fit forward model to imputations returned by mice }
\NormalTok{fit.fwd <-}\StringTok{ }\KeywordTok{with}\NormalTok{(}\DataTypeTok{data=}\NormalTok{imp_all, }\DataTypeTok{expr =} \KeywordTok{lm}\NormalTok{(}\DataTypeTok{formula =}\NormalTok{ wt }\OperatorTok{~}\StringTok{ }\NormalTok{gestation }\OperatorTok{+}\StringTok{ }
\StringTok{    }\NormalTok{smoke }\OperatorTok{+}\StringTok{ }\NormalTok{mht }\OperatorTok{+}\StringTok{ }\NormalTok{meth }\OperatorTok{+}\StringTok{ }\NormalTok{parity }\OperatorTok{+}\StringTok{ }\NormalTok{number }\OperatorTok{+}\StringTok{ }\NormalTok{mwt }\OperatorTok{+}\StringTok{ }\NormalTok{time }\OperatorTok{+}\StringTok{ }\NormalTok{gestation}\OperatorTok{:}\NormalTok{number }\OperatorTok{+}\StringTok{ }
\StringTok{    }\NormalTok{gestation}\OperatorTok{:}\NormalTok{mwt }\OperatorTok{+}\StringTok{ }\NormalTok{gestation}\OperatorTok{:}\NormalTok{mht))}


\CommentTok{# Compare stepwise and forward models with Cross Validation}
\NormalTok{nreps <-}\StringTok{ }\DecValTok{1000} \CommentTok{# number of replications}
\NormalTok{ntot <-}\StringTok{ }\KeywordTok{nrow}\NormalTok{(birthsCompl) }\CommentTok{# total number of observations}
\NormalTok{ntrain <-}\StringTok{ }\FloatTok{0.8}\OperatorTok{*}\NormalTok{ntot }\CommentTok{# size of training set is 80% of total }
\NormalTok{ntest <-}\StringTok{ }\NormalTok{ntot }\OperatorTok{-}\StringTok{ }\NormalTok{ntrain }\CommentTok{# size of test set is 20% of total }
\NormalTok{mspe1 <-}\StringTok{ }\KeywordTok{rep}\NormalTok{(}\OtherTok{NA}\NormalTok{, nreps) }\CommentTok{# sum-of-square errors for each CV replication}
\NormalTok{mspe2 <-}\StringTok{ }\KeywordTok{rep}\NormalTok{(}\OtherTok{NA}\NormalTok{, nreps)}

\NormalTok{logLambda <-}\StringTok{ }\KeywordTok{rep}\NormalTok{(}\OtherTok{NA}\NormalTok{, nreps) }\CommentTok{# log-likelihod ratio statistic for each replication}
\NormalTok{m <-}\DecValTok{5} \CommentTok{# number of imputations}
 

\ControlFlowTok{for}\NormalTok{ (ii }\ControlFlowTok{in} \DecValTok{1}\OperatorTok{:}\NormalTok{nreps) \{}
  \CommentTok{# randomly select training observations}
\NormalTok{  sample.ind <-}\StringTok{ }\KeywordTok{sample}\NormalTok{(m,}\DecValTok{1}\NormalTok{) }\CommentTok{# randomly select which imputed data set to use}
\NormalTok{  M1.model <-}\StringTok{ }\NormalTok{fit.fwd}\OperatorTok{$}\NormalTok{analyses[[sample.ind]]}
\NormalTok{  M2.model <-}\StringTok{ }\NormalTok{fit.step}\OperatorTok{$}\NormalTok{analyses[[sample.ind]]}
\NormalTok{  train.ind <-}\StringTok{ }\KeywordTok{sample}\NormalTok{(ntot, ntrain) }\CommentTok{# training observations}
  \CommentTok{# refit the models on the given imputed data and training indices }
\NormalTok{  data_complete <-}\StringTok{ }\KeywordTok{complete}\NormalTok{(imp_all,sample.ind)}
\NormalTok{  M1.cv <-}\StringTok{ }\KeywordTok{update}\NormalTok{(M1.model, }\DataTypeTok{data =}\NormalTok{ data_complete[train.ind,])}
\NormalTok{  M2.cv <-}\StringTok{ }\KeywordTok{update}\NormalTok{(M2.model, }\DataTypeTok{data =}\NormalTok{ data_complete[train.ind,])}
  \CommentTok{# out-of-sample residuals for both models}
  \CommentTok{# that is, testing data - predictions with training parameters}
\NormalTok{  M1.res <-}\StringTok{ }\NormalTok{data_complete}\OperatorTok{$}\NormalTok{wt[}\OperatorTok{-}\NormalTok{train.ind] }\OperatorTok{-}
\StringTok{    }\KeywordTok{predict}\NormalTok{(M1.cv, }\DataTypeTok{newdata =}\NormalTok{ data_complete[}\OperatorTok{-}\NormalTok{train.ind, ])}
\NormalTok{  M2.res <-}\StringTok{ }\NormalTok{data_complete}\OperatorTok{$}\NormalTok{wt[}\OperatorTok{-}\NormalTok{train.ind] }\OperatorTok{-}
\StringTok{    }\KeywordTok{predict}\NormalTok{(M2.cv, }\DataTypeTok{newdata =}\NormalTok{ data_complete[}\OperatorTok{-}\NormalTok{train.ind, ])}
  \CommentTok{# mean-square prediction errors}
\NormalTok{  mspe1[ii] <-}\StringTok{ }\KeywordTok{mean}\NormalTok{(M1.res }\OperatorTok{^}\StringTok{ }\DecValTok{2}\NormalTok{)}
\NormalTok{  mspe2[ii] <-}\StringTok{ }\KeywordTok{mean}\NormalTok{(M2.res }\OperatorTok{^}\StringTok{ }\DecValTok{2}\NormalTok{)}
  \CommentTok{# out-of-sample likelihood ratio}
\NormalTok{  M1.sigma <-}\StringTok{ }\KeywordTok{sqrt}\NormalTok{(}\KeywordTok{sum}\NormalTok{(}\KeywordTok{resid}\NormalTok{(M1.cv) }\OperatorTok{^}\StringTok{ }\DecValTok{2}\NormalTok{) }\OperatorTok{/}\StringTok{ }\NormalTok{ntrain) }\CommentTok{# MLE of sigma}
\NormalTok{  M2.sigma <-}\StringTok{ }\KeywordTok{sqrt}\NormalTok{(}\KeywordTok{sum}\NormalTok{(}\KeywordTok{resid}\NormalTok{(M2.cv) }\OperatorTok{^}\StringTok{ }\DecValTok{2}\NormalTok{) }\OperatorTok{/}\StringTok{ }\NormalTok{ntrain)}
  \CommentTok{# since res = y - pred, dnorm(y, pred, sd) = dnorm(res, 0, sd)}
\NormalTok{  logLambda[ii] <-}
\StringTok{    }\KeywordTok{sum}\NormalTok{(}\KeywordTok{dnorm}\NormalTok{(}
\NormalTok{      M1.res,}
      \DataTypeTok{mean =} \DecValTok{0}\NormalTok{,}
      \DataTypeTok{sd =}\NormalTok{ M1.sigma,}
      \DataTypeTok{log =} \OtherTok{TRUE}
\NormalTok{    ))}
\NormalTok{  logLambda[ii] <-}\StringTok{ }\NormalTok{logLambda[ii] }\OperatorTok{-}
\StringTok{    }\KeywordTok{sum}\NormalTok{(}\KeywordTok{dnorm}\NormalTok{(}
\NormalTok{      M2.res,}
      \DataTypeTok{mean =} \DecValTok{0}\NormalTok{,}
      \DataTypeTok{sd =}\NormalTok{ M2.sigma,}
      \DataTypeTok{log =} \OtherTok{TRUE}
\NormalTok{    ))}
\NormalTok{\}}

\CommentTok{# names of x axis for boxplot }
\NormalTok{Mnames <-}\StringTok{ }\KeywordTok{expression}\NormalTok{(M[FWD],M[STEP])}

\CommentTok{# Create boxplot}
\KeywordTok{par}\NormalTok{(}\DataTypeTok{mfrow=}\KeywordTok{c}\NormalTok{(}\DecValTok{1}\NormalTok{,}\DecValTok{2}\NormalTok{))}
\KeywordTok{boxplot}\NormalTok{(}\KeywordTok{list}\NormalTok{(}\KeywordTok{sqrt}\NormalTok{(mspe1),}\KeywordTok{sqrt}\NormalTok{(mspe2)), }\DataTypeTok{names =}\NormalTok{ Mnames, }\DataTypeTok{ylab =} \KeywordTok{expression}\NormalTok{(}\KeywordTok{sqrt}\NormalTok{(MSPE)), }\DataTypeTok{cex =} \FloatTok{0.7}\NormalTok{, }\DataTypeTok{col=} \KeywordTok{c}\NormalTok{(}\StringTok{"springgreen4"}\NormalTok{,}\StringTok{"skyblue"}\NormalTok{), }\DataTypeTok{cex.axis =} \FloatTok{0.7}\NormalTok{, }\DataTypeTok{main =} \StringTok{"RMSPE"}\NormalTok{)}

\CommentTok{# logLambda histgram}
\KeywordTok{hist}\NormalTok{(logLambda, }\DataTypeTok{breaks =} \DecValTok{30}\NormalTok{, }\DataTypeTok{freq =} \OtherTok{FALSE}\NormalTok{, }\DataTypeTok{xlab =} \KeywordTok{expression}\NormalTok{(Lambda}\OperatorTok{^}\NormalTok{\{test\}), }\DataTypeTok{cex=}\FloatTok{0.7}\NormalTok{,}
     \DataTypeTok{main =} \StringTok{"log of Lambda"}\NormalTok{, }\DataTypeTok{col=}\StringTok{"gray93"}\NormalTok{, }\DataTypeTok{cex.axis =} \FloatTok{0.7}\NormalTok{)}
\KeywordTok{abline}\NormalTok{(}\DataTypeTok{v=}\KeywordTok{mean}\NormalTok{(logLambda),}\DataTypeTok{col=}\StringTok{"red"}\NormalTok{,}\DataTypeTok{lwd=}\DecValTok{2}\NormalTok{)}
\end{Highlighting}
\end{Shaded}

\subsubsection{E. Model Diagnostics}\label{e.-model-diagnostics}

The following code looks at various residual plots for our two models.
This code generates Figures 6-9.

\begin{Shaded}
\begin{Highlighting}[]
\CommentTok{# We looked at residual plots for all 5 imputations for both models, but noticed that}
\NormalTok{## they were all very similar. As such, in the report and appendix we will only show }
\NormalTok{## the plots generated for the 5th imputation. }

\NormalTok{im <-}\StringTok{ }\DecValTok{5} \CommentTok{# let's just look at 5th imputation for both models}

\CommentTok{# Residuals for stepwise model}
\NormalTok{res.s <-}\StringTok{ }\KeywordTok{resid}\NormalTok{(fit.step}\OperatorTok{$}\NormalTok{analyses[[im]])}
\NormalTok{h.s <-}\StringTok{ }\KeywordTok{hatvalues}\NormalTok{(fit.step}\OperatorTok{$}\NormalTok{analyses[[im]])}
\NormalTok{res.stu.s <-}\StringTok{ }\NormalTok{res.s}\OperatorTok{/}\KeywordTok{sqrt}\NormalTok{(}\DecValTok{1}\OperatorTok{-}\NormalTok{h.s) }\CommentTok{# studentized residuals}

\CommentTok{# Residuals for forward model}
\NormalTok{res.f <-}\StringTok{ }\KeywordTok{resid}\NormalTok{(fit.fwd}\OperatorTok{$}\NormalTok{analyses[[im]])}
\NormalTok{h.f <-}\StringTok{ }\KeywordTok{hatvalues}\NormalTok{(fit.fwd}\OperatorTok{$}\NormalTok{analyses[[im]])}
\NormalTok{res.stu.f <-}\StringTok{ }\NormalTok{res.f}\OperatorTok{/}\KeywordTok{sqrt}\NormalTok{(}\DecValTok{1}\OperatorTok{-}\NormalTok{h.f) }\CommentTok{# studentized residuals}

\CommentTok{# Plotting residual plots for stepwise model, similar code for forwards model}
\KeywordTok{par}\NormalTok{(}\DataTypeTok{mfrow=}\KeywordTok{c}\NormalTok{(}\DecValTok{2}\NormalTok{,}\DecValTok{2}\NormalTok{))}
\KeywordTok{plot}\NormalTok{(fit.step}\OperatorTok{$}\NormalTok{analyses[[im]],}\DataTypeTok{cex=}\FloatTok{0.6}\NormalTok{,}\DataTypeTok{col=}\KeywordTok{adjustcolor}\NormalTok{(}\StringTok{"skyblue4"}\NormalTok{,}\FloatTok{0.5}\NormalTok{),}
     \DataTypeTok{cex.axis=}\FloatTok{0.7}\NormalTok{, }\DataTypeTok{pch=}\DecValTok{16}\NormalTok{)}

\CommentTok{# Plotting standardized residuals to see distribution, similar code for forwards model}
\KeywordTok{par}\NormalTok{(}\DataTypeTok{mfrow=}\KeywordTok{c}\NormalTok{(}\DecValTok{1}\NormalTok{,}\DecValTok{2}\NormalTok{))}
\NormalTok{sigma.hat.s <-}\StringTok{ }\KeywordTok{sigma}\NormalTok{(fit.step}\OperatorTok{$}\NormalTok{ana[[im]]) }\CommentTok{# pull RSE of 5th imputation}

\CommentTok{# Stepwise model studentized residual histogram}
\KeywordTok{hist}\NormalTok{(res.stu.s}\OperatorTok{/}\NormalTok{sigma.hat.s,}\DataTypeTok{breaks=}\DecValTok{40}\NormalTok{,}\DataTypeTok{freq=}\OtherTok{FALSE}\NormalTok{,}\DataTypeTok{cex.axis=}\FloatTok{0.8}\NormalTok{, }
     \DataTypeTok{main =} \StringTok{"Stepwise Model"}\NormalTok{, }\DataTypeTok{xlab=} \StringTok{"Studentized Residuals, standardized"}\NormalTok{)}
\KeywordTok{curve}\NormalTok{(}\KeywordTok{dnorm}\NormalTok{(x),}\DataTypeTok{col=}\StringTok{"red"}\NormalTok{,}\DataTypeTok{add=}\OtherTok{TRUE}\NormalTok{) }\CommentTok{# normal dist overlay}

\CommentTok{# Create influence vs. leverage plot with each model on the same graph}
\CommentTok{# Calculate measures of influence - Cook's distance for each model}
\NormalTok{D.step <-}\StringTok{ }\KeywordTok{cooks.distance}\NormalTok{(fit.step}\OperatorTok{$}\NormalTok{analyses[[im]])}
\NormalTok{D.fwd <-}\StringTok{ }\KeywordTok{cooks.distance}\NormalTok{(fit.fwd}\OperatorTok{$}\NormalTok{analyses[[im]])}
\CommentTok{# Calculate diagonals of hat matrix to get leverage measures for each model}
\NormalTok{h.step <-}\StringTok{ }\KeywordTok{hatvalues}\NormalTok{(fit.step}\OperatorTok{$}\NormalTok{analyses[[im]])}
\NormalTok{h.fwd <-}\StringTok{ }\KeywordTok{hatvalues}\NormalTok{(fit.fwd}\OperatorTok{$}\NormalTok{analyses[[im]])}

\NormalTok{inf <-}\StringTok{ }\KeywordTok{which.is.max}\NormalTok{(D.step) }\CommentTok{# pull index of highest influence observation }
\NormalTok{lev <-}\StringTok{ }\KeywordTok{which.is.max}\NormalTok{(h.step) }\CommentTok{# pull index of highest leverage observation }


\CommentTok{# Create scatterplot}
\CommentTok{# First add red points for stepwise model}
\KeywordTok{plot}\NormalTok{(h.step,D.step, }\DataTypeTok{col=}\KeywordTok{adjustcolor}\NormalTok{(}\StringTok{"firebrick"}\NormalTok{,}\FloatTok{0.5}\NormalTok{), }\DataTypeTok{pch=}\DecValTok{16}\NormalTok{, }
     \DataTypeTok{main =} \StringTok{"Influence vs. Leverage Plot"}\NormalTok{, }\DataTypeTok{xlab =} \StringTok{"Leverage"}\NormalTok{, }
     \DataTypeTok{ylab=} \StringTok{"Influence (Cook's Distance)"}\NormalTok{, }\DataTypeTok{cex.axis=}\FloatTok{0.8}\NormalTok{)}
\CommentTok{# Now add blue points for forwards model}
\KeywordTok{points}\NormalTok{(h.fwd,D.fwd,}\DataTypeTok{col=}\KeywordTok{adjustcolor}\NormalTok{(}\StringTok{"navyblue"}\NormalTok{,}\FloatTok{0.5}\NormalTok{),}\DataTypeTok{pch=}\DecValTok{18}\NormalTok{)}
\CommentTok{# Throw in a legend }
\KeywordTok{legend}\NormalTok{(}\StringTok{"topright"}\NormalTok{, }\DataTypeTok{legend =} \KeywordTok{c}\NormalTok{(}\StringTok{"Stepwise"}\NormalTok{, }\StringTok{"Forward"}\NormalTok{), }\DataTypeTok{pch =} \KeywordTok{c}\NormalTok{(}\DecValTok{16}\NormalTok{,}\DecValTok{18}\NormalTok{),}
       \DataTypeTok{col =} \KeywordTok{c}\NormalTok{(}\StringTok{"firebrick"}\NormalTok{,}\StringTok{"navyblue"}\NormalTok{))}
\CommentTok{# Label highest influence point}
\KeywordTok{text}\NormalTok{(}\DataTypeTok{x=}\NormalTok{h.step[inf], }\DataTypeTok{y =}\NormalTok{ D.step[inf],}
     \DataTypeTok{labels =}\NormalTok{ inf, }\DataTypeTok{pos=}\DecValTok{1}\NormalTok{, }\DataTypeTok{cex =} \FloatTok{0.7}\NormalTok{, }\DataTypeTok{offs=}\FloatTok{0.4}\NormalTok{)}
\CommentTok{# Label highest leverage point}
\KeywordTok{text}\NormalTok{(}\DataTypeTok{x=}\NormalTok{h.step[lev], }\DataTypeTok{y =}\NormalTok{ D.step[lev],}
     \DataTypeTok{labels =}\NormalTok{ lev, }\DataTypeTok{pos=}\KeywordTok{c}\NormalTok{(}\DecValTok{1}\NormalTok{,}\DecValTok{1}\NormalTok{,}\DecValTok{1}\NormalTok{), }\DataTypeTok{cex =} \FloatTok{0.7}\NormalTok{, }\DataTypeTok{offs=}\FloatTok{0.4}\NormalTok{)}

\CommentTok{# Pulling out highest influence point observation to take a look at }
\KeywordTok{complete}\NormalTok{(imp_all,}\DecValTok{5}\NormalTok{)[inf.ind,]}

\CommentTok{# Pulling out highest leverage points observation to take a look at }
\KeywordTok{complete}\NormalTok{(imp_all,}\DecValTok{5}\NormalTok{)[lev.ind,]}
\end{Highlighting}
\end{Shaded}

\subsubsection{F. Final Model Tuning}\label{f.-final-model-tuning}

First, we re-ran automatic stepwise selection on the data set with the
outlier removed.

\begin{Shaded}
\begin{Highlighting}[]
\ControlFlowTok{for}\NormalTok{ (i }\ControlFlowTok{in} \DecValTok{1}\OperatorTok{:}\DecValTok{5}\NormalTok{)\{}
\NormalTok{  compldata <-}\StringTok{ }\KeywordTok{complete}\NormalTok{(imp_all,i)[}\OperatorTok{-}\DecValTok{246}\NormalTok{,] }\CommentTok{# Completing data set over each imputation}
\NormalTok{  M0.out <-}\StringTok{ }\KeywordTok{lm}\NormalTok{(wt}\OperatorTok{~}\DecValTok{1}\NormalTok{,}\DataTypeTok{data=}\NormalTok{compldata) }\CommentTok{# minimal model: intercept only}
  
  \CommentTok{# all main effects and interactions}
\NormalTok{  ##  Removed feth, meth, fed, med interactions because they produced NA's}
\NormalTok{  Mmax.out <-}\StringTok{ }\KeywordTok{lm}\NormalTok{(wt }\OperatorTok{~}\StringTok{ }\NormalTok{(.}\OperatorTok{-}\NormalTok{feth }\OperatorTok{-}\StringTok{ }\NormalTok{meth }\OperatorTok{-}\StringTok{ }\NormalTok{fed }\OperatorTok{-}\StringTok{ }\NormalTok{med)}\OperatorTok{^}\DecValTok{2} \OperatorTok{+}\StringTok{ }\NormalTok{feth }\OperatorTok{+}\StringTok{ }\NormalTok{meth }\OperatorTok{+}\StringTok{ }\NormalTok{fed }\OperatorTok{+}\StringTok{ }\NormalTok{med,}\DataTypeTok{data=}\NormalTok{compldata)}
\NormalTok{  Mstart.out <-}\KeywordTok{lm}\NormalTok{(wt }\OperatorTok{~}\NormalTok{., }\DataTypeTok{data =}\NormalTok{ compldata) }\CommentTok{# starting model for stepwise}
\NormalTok{  Mstep1 <-}\StringTok{ }\KeywordTok{step}\NormalTok{(}\DataTypeTok{object=}\NormalTok{ Mstart.out,}
              \DataTypeTok{scope=}\KeywordTok{list}\NormalTok{(}\DataTypeTok{lower=}\NormalTok{M0.out, }\DataTypeTok{upper =}\NormalTok{ Mmax.out),}
              \DataTypeTok{direction =}\StringTok{"both"}\NormalTok{,}
              \DataTypeTok{trace=}\OtherTok{FALSE}\NormalTok{)}
\NormalTok{  model <-}\StringTok{ }\KeywordTok{paste}\NormalTok{(}\StringTok{"MsOut"}\NormalTok{,i, }\DataTypeTok{sep=}\StringTok{""}\NormalTok{) }\CommentTok{# dynamically name models with corresponding index}
  \KeywordTok{assign}\NormalTok{(model,Mstep1)}
  \KeywordTok{print}\NormalTok{(Mstep1}\OperatorTok{$}\NormalTok{call)}
\NormalTok{\}}
\end{Highlighting}
\end{Shaded}

Then re-ran cross-validation on our tweaked stepwise model and the
original one. The code belows shows the results of this, and how they
turned out indistinguishable.

\begin{Shaded}
\begin{Highlighting}[]
\KeywordTok{options}\NormalTok{(}\DataTypeTok{warn=}\OperatorTok{-}\DecValTok{1}\NormalTok{)}



\KeywordTok{set.seed}\NormalTok{(}\DecValTok{331}\NormalTok{)}
 \CommentTok{# Cross-validation setup}
\NormalTok{ nreps <-}\StringTok{ }\DecValTok{1000} \CommentTok{# number of replications}
\NormalTok{ ntot <-}\StringTok{ }\KeywordTok{nrow}\NormalTok{(birthsCompl) }\CommentTok{# total number of observations}
\NormalTok{ ntrain <-}\StringTok{ }\FloatTok{0.8}\OperatorTok{*}\NormalTok{ntot }\CommentTok{# size of training set is 80% of total }
\NormalTok{ ntest <-}\StringTok{ }\NormalTok{ntot }\OperatorTok{-}\StringTok{ }\NormalTok{ntrain }\CommentTok{# size of test set is 20% of total }
\NormalTok{ mspe1 <-}\StringTok{ }\KeywordTok{rep}\NormalTok{(}\OtherTok{NA}\NormalTok{, nreps) }\CommentTok{# sum-of-square errors for each CV replication}
\NormalTok{ mspe2 <-}\StringTok{ }\KeywordTok{rep}\NormalTok{(}\OtherTok{NA}\NormalTok{, nreps)}

\NormalTok{ logLambda <-}\StringTok{ }\KeywordTok{rep}\NormalTok{(}\OtherTok{NA}\NormalTok{, nreps) }\CommentTok{# log-likelihod ratio statistic for each replication}
\NormalTok{ m <-}\DecValTok{5} \CommentTok{# number of imputations}
 

 \ControlFlowTok{for}\NormalTok{ (ii }\ControlFlowTok{in} \DecValTok{1}\OperatorTok{:}\NormalTok{nreps) \{}
   
   \CommentTok{# randomly select training observations}
\NormalTok{   sample.ind <-}\StringTok{ }\KeywordTok{sample}\NormalTok{(m,}\DecValTok{1}\NormalTok{) }\CommentTok{# randomly select which imputed data set to use}
   
\NormalTok{   M1.model <-}\StringTok{ }\NormalTok{fit.stepOut}\OperatorTok{$}\NormalTok{analyses[[sample.ind]]}
\NormalTok{   M2.model <-}\StringTok{ }\NormalTok{fit.step}\OperatorTok{$}\NormalTok{analyses[[sample.ind]]}
\NormalTok{   train.ind <-}\StringTok{ }\KeywordTok{sample}\NormalTok{(ntot, ntrain) }\CommentTok{# training observations}
   \CommentTok{# refit the models on the given imputed data and training indices }
\NormalTok{   data_complete <-}\StringTok{ }\KeywordTok{complete}\NormalTok{(imp_all,sample.ind)}
\NormalTok{   M1.cv <-}\StringTok{ }\KeywordTok{update}\NormalTok{(M1.model, }\DataTypeTok{data =}\NormalTok{ data_complete[train.ind,])}
\NormalTok{   M2.cv <-}\StringTok{ }\KeywordTok{update}\NormalTok{(M2.model, }\DataTypeTok{data =}\NormalTok{ data_complete[train.ind,])}
   \CommentTok{# out-of-sample residuals for both models}
   \CommentTok{# that is, testing data - predictions with training parameters}
\NormalTok{   M1.res <-}\StringTok{ }\NormalTok{data_complete}\OperatorTok{$}\NormalTok{wt[}\OperatorTok{-}\NormalTok{train.ind] }\OperatorTok{-}
\StringTok{     }\KeywordTok{predict}\NormalTok{(M1.cv, }\DataTypeTok{newdata =}\NormalTok{ data_complete[}\OperatorTok{-}\NormalTok{train.ind, ])}
\NormalTok{   M2.res <-}\StringTok{ }\NormalTok{data_complete}\OperatorTok{$}\NormalTok{wt[}\OperatorTok{-}\NormalTok{train.ind] }\OperatorTok{-}
\StringTok{     }\KeywordTok{predict}\NormalTok{(M2.cv, }\DataTypeTok{newdata =}\NormalTok{ data_complete[}\OperatorTok{-}\NormalTok{train.ind, ])}
   \CommentTok{# mean-square prediction errors}
\NormalTok{   mspe1[ii] <-}\StringTok{ }\KeywordTok{mean}\NormalTok{(M1.res }\OperatorTok{^}\StringTok{ }\DecValTok{2}\NormalTok{)}
\NormalTok{   mspe2[ii] <-}\StringTok{ }\KeywordTok{mean}\NormalTok{(M2.res }\OperatorTok{^}\StringTok{ }\DecValTok{2}\NormalTok{)}
   \CommentTok{# out-of-sample likelihood ratio}
\NormalTok{   M1.sigma <-}\StringTok{ }\KeywordTok{sqrt}\NormalTok{(}\KeywordTok{sum}\NormalTok{(}\KeywordTok{resid}\NormalTok{(M1.cv) }\OperatorTok{^}\StringTok{ }\DecValTok{2}\NormalTok{) }\OperatorTok{/}\StringTok{ }\NormalTok{ntrain) }\CommentTok{# MLE of sigma}
\NormalTok{   M2.sigma <-}\StringTok{ }\KeywordTok{sqrt}\NormalTok{(}\KeywordTok{sum}\NormalTok{(}\KeywordTok{resid}\NormalTok{(M2.cv) }\OperatorTok{^}\StringTok{ }\DecValTok{2}\NormalTok{) }\OperatorTok{/}\StringTok{ }\NormalTok{ntrain)}
   \CommentTok{# since res = y - pred, dnorm(y, pred, sd) = dnorm(res, 0, sd)}
\NormalTok{   logLambda[ii] <-}
\StringTok{     }\KeywordTok{sum}\NormalTok{(}\KeywordTok{dnorm}\NormalTok{(}
\NormalTok{       M1.res,}
       \DataTypeTok{mean =} \DecValTok{0}\NormalTok{,}
       \DataTypeTok{sd =}\NormalTok{ M1.sigma,}
       \DataTypeTok{log =} \OtherTok{TRUE}
\NormalTok{     ))}
\NormalTok{   logLambda[ii] <-}\StringTok{ }\NormalTok{logLambda[ii] }\OperatorTok{-}
\StringTok{     }\KeywordTok{sum}\NormalTok{(}\KeywordTok{dnorm}\NormalTok{(}
\NormalTok{       M2.res,}
       \DataTypeTok{mean =} \DecValTok{0}\NormalTok{,}
       \DataTypeTok{sd =}\NormalTok{ M2.sigma,}
       \DataTypeTok{log =} \OtherTok{TRUE}
\NormalTok{     ))}
\NormalTok{ \}}
 
\CommentTok{# names for x axis of boxplot }
\NormalTok{Mnames <-}\StringTok{ }\KeywordTok{expression}\NormalTok{(M[STEPout],M[STEP])}

\CommentTok{# create boxplot}
\KeywordTok{par}\NormalTok{(}\DataTypeTok{mfrow=}\KeywordTok{c}\NormalTok{(}\DecValTok{1}\NormalTok{,}\DecValTok{2}\NormalTok{))}
\KeywordTok{boxplot}\NormalTok{(}\KeywordTok{list}\NormalTok{(}\KeywordTok{sqrt}\NormalTok{(mspe1),}\KeywordTok{sqrt}\NormalTok{(mspe2)), }\DataTypeTok{names =}\NormalTok{ Mnames, }\DataTypeTok{ylab =} \KeywordTok{expression}\NormalTok{(}\KeywordTok{sqrt}\NormalTok{(MSPE)), }\DataTypeTok{cex =} \FloatTok{0.7}\NormalTok{, }\DataTypeTok{col=} \KeywordTok{c}\NormalTok{(}\StringTok{"springgreen4"}\NormalTok{,}\StringTok{"skyblue"}\NormalTok{), }\DataTypeTok{cex.axis =} \FloatTok{0.7}\NormalTok{, }\DataTypeTok{main =} \StringTok{"RMSPE"}\NormalTok{)}

\CommentTok{# logLambda histogram}
\KeywordTok{hist}\NormalTok{(logLambda, }\DataTypeTok{breaks =} \DecValTok{30}\NormalTok{, }\DataTypeTok{freq =} \OtherTok{FALSE}\NormalTok{, }\DataTypeTok{xlab =} \KeywordTok{expression}\NormalTok{(Lambda}\OperatorTok{^}\NormalTok{\{test\}), }\DataTypeTok{cex=}\FloatTok{0.7}\NormalTok{,}
     \DataTypeTok{main =} \StringTok{"log of Lambda"}\NormalTok{, }\DataTypeTok{col=}\StringTok{"gray93"}\NormalTok{, }\DataTypeTok{cex.axis =} \FloatTok{0.7}\NormalTok{)}
\KeywordTok{abline}\NormalTok{(}\DataTypeTok{v=}\KeywordTok{mean}\NormalTok{(logLambda),}\DataTypeTok{col=}\StringTok{"red"}\NormalTok{,}\DataTypeTok{lwd=}\DecValTok{2}\NormalTok{)}
\end{Highlighting}
\end{Shaded}

\includegraphics{Final_WriteUp_files/figure-latex/unnamed-chunk-30-1.pdf}

\subsection{Bibliography}\label{bibliography}

@journal\{LBWarticle1, title = ``The Effects Of Maternal Smoking,
\ldots{}, On The Incidence Of Low Birth Weight'', author = ``Kleinmann,
J and Madans, J.'', month = ``February'', year = ``2016'', url =
``\url{http://citeseerx.ist.psu.edu/viewdoc/download?doi=10.1.1.820.7552\&rep=rep1\&type=pdf}''
\}

@journal\{BWarticle2, title = ``Quantitative Data Analysis of Multiple
Factors Associated with Low Birth Weight in Bibb County, Georgia'',
author = ``Jackson, H., Wei, Y., Chen, F.'', year = \{2008\}, url =
``\url{https://augusta.openrepository.com/augusta/bitstream/10675.2/610914/1/Jackson_2008_1_1.pdf}'',
\}


\end{document}
